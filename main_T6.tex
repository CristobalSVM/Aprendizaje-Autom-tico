\documentclass{article}
\usepackage{graphicx} % Required for inserting images

\usepackage[spanish]{babel}
\usepackage[T1]{fontenc}
\usepackage{natbib}

\title{Predicción de Activos financieros en función de los pasivos más relevantes de la institución financiera Banregio Grupo Financiero I.B.M, mediante regresión lineal}

\author{Cristóbal Salinas \\ 
        Facultad de Ciencias Físico-Matemáticas, \\Universidad Autónoma de Nuevo León, \\San Nicolás de los Garza, Nuevo León}

\date{\today}

\begin{document}

\maketitle

\begin{abstract}

En la presente investigación se busca predecir de la institución financiera Banregio Grupo Financiero I.B.M., los activos que se generan cada mes en función de los depósitos de los clientes y préstamos interbancarios, complementando también con el número de sucursales, número de tarjetas de crédito y número de cajeros. \\

El modelo seleccionado para el análisis fue de regresión para ajustar una curva a los datos minimizando el error y se hizo una predicción para el mes de enero 2022.

\textit{Palabras clave: Activo, vista, plazo, préstamos interbancarios, regresión líneal.}

\end{abstract}

\section{Introduction}

En el tema de Finanzas, se tiene la fórmula donde los activos son iguales a los pasivos más capital contable, nos enfocaremos los pasivos en particular los depósitos a la vista y plazo de los clientes y prestamos interbancarios por ser los rubros de mayor relevancia para Banregio y en particular, para las bancas comerciales como lo son los bancos tradicionales que conocemos como Afirme, Banorte, BBVA, entre otros, ya que con estos depósitos se pueden hacer préstamos a clientes o inversiones a plazo que generen intereses a favor del banco y puedan generar más activos.
\\
\\
Los activos también varían en función de la cantidad de sucursales, tarjetas de crédito y cajeros automáticos, ya que con estos se puede tener una mayor atracción de clientes que depósiten sus ahorros en el banco de su preferencia.
\\
\\
Por definición, los activos siempre deben ser mayores que los pasivos para evitar problemas de liquidez del banco y evitar afectar tanto a los clientes como al sistema financiero y economico del país, como menciona \citet{mu2021}, la banca pública y privada, como medida precautoria, tendrá que ocupar los montos de sus activos líquidos (fondos disponibles e inversiones) para hacer frente a sus pasivos.
\\
\\
Según \citet{san12} Las investigaciones en la predicción de valores futuros de activos financieros, han venido desarrollandose en base a comportamientos anteriores de los mismos casos de estudio, usandose herramientas estadísticas, obteniendose resultados muy útiles en la mayoría de los casos.



\subsection{Objetivo}

Predecir los activos de Banregio con la mayor precisión posible de acuerdo a la información observada de la parte pasiva relevante que son los depósitos a la vista, depósitos a plazo, préstamos interbancarios en complemento con las tarjetas de crédito, sucursales y cajeros.


\section{Literatura relacionada}

Dentro de la literatura, se encuentran pocos trabajos relacionados referentes a la predicción de activos financieros, en particular, en la predicción de la cartera de crédito.
\\
\\
Un trabajo relacionado a este tema es el de \citet{romero2015}, en el cual propone un modelo econométrico (ARIMA) que permite proyectar o predecir el comportamiento de variables, como lo es el activo más significativo de dicho tipo de entidades, que es la cartera de créditos referente a las cooperativas de ahorro y crédito.
\\
\\
También \citet{duque2012} menciona que utilizar variaciones y regresiones de datos estadísticos es con el propósito de realizar proyecciones hacia el futuro de activos, pasivos, cartera, patrimonio y captaciones, en este trabajo, selecciona los balances
consolidados de las Cooperativas de Ahorro y Crédito reguladas por la Superintendencia
de Bancos y Seguros publicados en su página web.



\section{Marco teórico}



\section{Metodología}

Para este estudio, se considera el periodo de diciembre 2016 a diciembre 2021 y la información financiera del total de la parte activa y de los depósitos en vista y plazo, préstamos interbancarios, así como la cantidad de sucursales, cajeros y tarjetas de crédito de la institución de banca múltiple Banregio. La información tomada es de los estados financieros y se publica mensualmente en la página de la Comisión Nacional Bancaria y de Valores.
\\

Se utilizó el modelo DBSCAN a los datos ya que este el algoritmo trabaja con datos con formas extrañas, además, por ser no supervisado, da un poco más de flexibilidad de elegir los parámetros iniciales, los cuales podrían definirse por ejemplo, min sample = 3 por ser datos financieros, Banxico y la CNBV piden información trimestral y eps dependerá de la estrategia de negocio. En este caso, Banregio se concentra en personas físicas, personas físicas con actividad empresarial y MyPYMES, por lo que se podría decir que un eps de 0.275 MDP es una buena aproximación de acuerdo a los estados financieros.
\\

Para confirmar el mínimo de observaciones para formar un cluster es de tres, se aplico el método del codo, calculando la distancia entre cada punto y la recta del del punto inicial al punto final, tomando como criterio la primer distancia mayor.
\\

Se aplicó también el método de regresión lineal múltiple para predecir los valores de la variable independiente (Activo) del modelo.
\\

En una regresión se busca ajustar una curva a los datos minimizando el error. La regresión más sencilla es la regresión lineal donde se pretende predecir valores $Y$ a partir de determinados $n$ variables mediante la ecuación lineal $Y = b_0 + X_1b_1 + ... + X_nb_n$, donde $b_0$ coincide con una constante o intercepción, mientras que $b_i,i \epsilon {1,2,...,n}$ son la pendiente para cada $X$.


\section{Resultados}

La siguiente figura muestra el uso del método de codo para selección de número de grupos.
\\

\begin{figure}
    \centering
    \includegraphics[width=1\textwidth]{figs/codo.pdf}
    \caption{Selección de número de grupos}
    \label{fig:Método de codo}
\end{figure}

\begin{figure}
    \centering
    \includegraphics[width=1\textwidth]{figs/dbscan.pdf}
    \caption{Selección de número de grupos}
    \label{fig:Método de codo}
\end{figure}

\\

Se observa que la primer mayor distancia es de 165.69, por lo tanto, se confirma que el número de grupos adecuados es 3.
\\

En la figura 2 se puede ver el agrupamiento de los 3 cluster, considerando el eje y como activos en función del eje x como la suma del resto de las variables.
\\

Al aplicar un segundo método, Regresión Lineal Múltiple, se calculo el Error Porcentual Absoluto Medio (MAPE por sus siglas en ingles) obteniendo de resultado 0.0167 y al gráficar los activos originales con sus respectivas predicciones se obtuvo la gráfica de la figura 3.
\\

\begin{figure}
    \centering
    \includegraphics[width=1\textwidth]{figs/reg.pdf}
    \caption{Comparación del activo original y sus predicciones}
    \label{fig:Método de codo}
\end{figure}
\\

\section{Discusión}

Los resultados obtenidos aplicando el método DBSCAN coincide con lo esperado, es decir, 
 el grupo adecuado que sugiere este método es de 3, como se sospechaba de un inicio por la medición trimestral que señala la regulación bancaria al revelar información en esa periodicidad.
\\

 Al revisar los clusters, se muestran los 3 grupos, se esperaba que estos tuvieran aproximadamente la misma cantidad de puntos, ya que, se tiene cierta estacionalidad cada trimestre por el tipo de clientes que tienen las mismas caracteristicas entre sí para poder tener una mayor certeza en su comportamiento de los depósitos, pero no fue así.
\\

 De acuerdo con el método de Regresión Lineal Múltiple, la gráfica de las predicciones va en contra de la teoría, ya que esta no es una línea recta.

\section{Conclusiones}
DBSCAN no es un buen método para agrupar los datos de este estudio ya que se tienen factores que no necesariamente se tienen que tratar con este modelo, en el caso del método de codo, cla cantidad de grupos arrojada es adecuada ya que es un dato que se tiene que cumplir por ser regulatorio.
\\

Al hacer uso de la Regresión Lineal Múltiple, se tiene un buen ajuste del modelo de acuerdo con el calculo del MAPE, pero no hay que basarnos solo en esta medición, también hay que complementar con la gráfica del modelo.  
\\

En la gráfica de los valores originales y las predicciones de los activos, se observa la misma tendencia a la alta y en varios puntos coincide pero por teoría, la línea negra debería ser una línea recta, por lo tanto, y aunque coincida en varios puntos, no es un modelo que ajuste de forma adecuada a estos datos.


\section{Trabajo a futuro}
Aplicar otros métodos para encontrar mejores predicciones y un adecuado tratamiento de los datos.
\\

Una sugerencia de modelo podría ser el método XGBoost, como menciona \cite{zuñ2020} conviene utilizarlo porque puede manejar grandes bases de datos con múltiples variables, valores perdidos, sus resultados son muy precisos y es muy bueno en velocidad de ejecución y también trabaja con valores númericos y Las ventajas de este algoritmo hace que se aplique en campos como: identificación de huellas digitales, seguridad vial y análisis de mercados financieros, entre otros.


\bibliographystyle{unsrtnat}
\bibliography{biblio}

\end{document}

